\section{The CS TA Grading Problem}
\label{sec:problem}

\tab A predominant issue that arises in programming courses is the process of grading homeworks.
Usually these courses make use of one or multiple teaching assistants to manage
the grading process, given that the time and effort required may not be feasible by the
professor himself.
I have been a teaching assistant for several graduate courses in EECS department at UTK.
In Spring 2016 I was TA for \textit{``COSC 505 - Introduction to Programming
for Scientists and Engineers''} taught by Dr. Gregory D. Peterson.
It was the first time as a TA in which I had to grade over 30 programming homeworks
in a weekly manner.
Lets consider that it takes 10 minutes per homework,
including submitting the student's grade and feedback,
then it would take 5 hours for all homeworks which amounts to 25\% -- 50\% of a TA's time
considering a 10 -- 20 hour appointment.
To be more realistic, the time per homework varies based on the problem's difficulty,
speed of user inputs, and code structure,
leading to cases where it takes noticeably more than 10 minutes.
The total number of students enrolled in introductory programming courses at UTK
surpasses 30 students by much (multiple courses, multiple sections) which means
that a large portion of TA's time is spent grading.
Reducing the grading process would increase TA's availability for office hours, tutoring,
and research work. \par
\vspace{1em}

From my experience, the grading process involves the following steps:
\begin{enumerate}
    \item Students submit homeworks via a method established by the professor
    \begin{enumerate}
        \item course delivery and management software (Blackboard, Canvas)
        \item server associated with institution/deparment
        \item other internet technologies (file sharing, repositories, email) 
    \end{enumerate}
    \item TA acquires homeworks
    \item TA processes homeworks
    \item TA provides grades and feedback to students, usually via the same
          method used for homework submission
\end{enumerate}

\tab Steps 1 and 2 can be accomplished easily and efficiently with course management software.
If another method is used for homework submissions, an important aspect to specify 
is a naming convention for the submission package
to prevent name collisions, overwrites, or human confusion.
Step 3 is the core step and has several inherent issues.
Most programs make use of some type of input (e.g., user, file, another program) and
the search space between correct inputs-outputs can be infinite.
Also, depending on the course instructions, a submission package may include
additional files besides the source files (e.g., manual, reports, figures).
Currently, there is no single solution for resolving these issues.
\hyperref[sec:litreview]{Section 3} discusses several existing tools and approaches.
Step 4 depends on step 1 for the method used to interface with students.
Nevertheless, most likely the TA has to upload both the grades and feedback
for each student one at a time.


